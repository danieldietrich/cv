%% Copyright 2006-1008 Xavier Danaux (xdanaux@gmail.com).
%
% This work may be distributed and/or modified under the
% conditions of the LaTeX Project Public License version 1.3c,
% available at http://www.latex-project.org/lppl/.

\documentclass[10pt,a4paper,colorlinks=true]{moderncv}

%----------------------------------------------------------------------------------
%  moderncv theme
%----------------------------------------------------------------------------------

% optional arguments are 'blue' (default), 'orange', 'red', 'green', 'grey' and 'roman' (for roman fonts, instead of sans serif fonts)
%\moderncvtheme[blue, roman]{casual}
\moderncvtheme[blue]{classic}

% Character encoding
\usepackage[utf8]{inputenc} % replace by the encoding you are using
\usepackage[english]{babel}

\usepackage{fontawesome}
\usepackage{changepage} % for the adjustwidth environment

% Add a timeline on the left, API see http://ctan.math.utah.edu/ctan/tex-archive/macros/latex/contrib/moderntimeline/moderntimeline.pdf
\usepackage[firstyear=1998,lastyear=2020]{moderntimeline}
\tlenablemonths
\setlength{\hintscolumnwidth}{4.1cm} % Width of the timeline on your left

% Globally modify itemsep on itemize environment
\usepackage{enumitem}
\setitemize{itemsep=4pt}

% Adjust the page margins
%\addtolength{\voffset}{1cm}
\usepackage[scale=0.8]{geometry}
%\setlength{\hintscolumnwidth}{3cm}                      % if you want to change the width of the column with the dates
%\AtBeginDocument{\setlength{\maketitlenamewidth}{6cm}}  % only for the classic theme, if you want to change the width of your name placeholder (to leave more space for your address details
\AtBeginDocument{\recomputelengths}                      % required when changes are made to page layout lengths

%\nopagenumbers{}                                        % uncomment to suppress automatic page numbering for CVs longer than one page

%----------------------------------------------------------------------------------
%  Fork me on GitHub
%----------------------------------------------------------------------------------

\usepackage{tikz}
\usepackage{ifthen}
\usepackage{xcolor}
\usetikzlibrary{shadows.blur}

\definecolor{forkmebg}{HTML}{3873b3}
\definecolor{forkmefg}{HTML}{eeeeee}

% rotate and shift are hardcoded here because of the href workaround, see also https://gist.github.com/dokenzy/fe8967c8a5921191a261
\newcommand{\forkme}{
  \begin{tikzpicture}[remember picture, overlay]
    \node[rotate=45, shift={(0, -3.2cm)}] at (current page.north west) {
      \begin{tikzpicture}[remember picture, overlay]
        \node[fill=forkmebg, text centered, minimum width=50em, minimum height=3.0em, blur shadow, shadow yshift=0pt, shadow xshift=0pt, shadow blur radius=.4em, shadow opacity=50](fmogh) at (0pt, 0pt) {}; % workaround: text/link see below
        \draw[forkmefg!60, dashed, line width=.08em, dash pattern=on .5em off 1.5\pgflinewidth] (-25em,1.2em) rectangle (25em,-1.2em);
      \end{tikzpicture}
    };
    % workaround: the link needs to be defined using a \rotatebox{deg} because \node[rotate=deg]{\href} isn't rendered correctly
    \node[shift={(0.8em, 1.7em)}, text centered, minimum width=50em, minimum height=3.0em, text=forkmefg](fmogh) at (0pt, 0pt) { \fontfamily{phv}\selectfont\bfseries \href{https://github.com/danieldietrich/cv}{\rotatebox{45}{\color{white}{Fork me on GitHub}}} };
  \end{tikzpicture}
}

%----------------------------------------------------------------------------------
%  Personal data
%----------------------------------------------------------------------------------

\firstname{Daniel}
\familyname{Dietrich}
\title{Curriculum Vitae}                    % optional, remove the line if not wanted
\address{\faMapMarker~ Kiel, Germany}       % optional, remove the line if not wanted
\mobile{+49 151 645 00 968}                 % optional, remove the line if not wanted
%\phone{}                                   % optional, remove the line if not wanted
%\fax{fax (optional)}                       % optional, remove the line if not wanted
\email{mail@danieldietrich.dev}             % optional, remove the line if not wanted
%\extrainfo{Nationality: German}            % optional, remove the line if not wanted
\photo[64pt]{portrait.jpg}                  % '64pt' is the height the picture must be resized to and 'picture' is the name of the picture file; optional, remove the line if not wanted
%\quote{Some quote (optional)}              % optional, remove the line if not wanted

%----------------------------------------------------------------------------------
%  Content
%----------------------------------------------------------------------------------

\begin{document}

\forkme

\maketitle

% Increase cventry vertical space
\setlength{\parskip}{4pt}

\begin{flushright}Last updated: \today \end{flushright}
[ \emph{Read the résumé version} at \href{https://resume.danieldietrich.dev}{https://resume.danieldietrich.dev} ]

\section{\textsc{Summary}}
\vspace{1em}
\begin{adjustwidth}{1cm}{1cm}
Daniel is a highly motivated Senior Software Engineer, Open Source Community Leader and Author with 18+ years of experience.
He is a skilled lead developer, software architect and project manager with distinct analytical skills.
His portfolio ranges from the design of web and user interfaces to middleware, infrastructure and databases.
\\[2pt]
Beside his regular job, Daniel is an active open source contributor and author.
Daniel is intrinsically motivated to learn new languages, frameworks and technologies.
He is sharing his knowledge by writing articles and books and collaborates with developers around the world on the GitHub platform.
His most visible open source project is Vavr, a library for object-functional programming in Java.
\\[2pt]
Daniel is a polyglot programmer and can switch between different paradigms, like functional, object-oriented and declarative (SQL, NoSQL) programming.
He uses the language that fits best.
During the last years, Daniel focused mainly on the web platform for frontend development, using React and TypeScript,
and he developed with Node.js and Java on the backend.
\\[2pt]
Out of curiosity, he studied the React source code by Facebook and developed his own React clone, including the modern Hooks API (only 2kb minzipped).
In another personal side-project, he explored the limits of TypeScript's type system by implementing an
internal DSL for CouchDB's Mango Query Language (in only 35 lines of code).
\\[2pt]
Daniel has a creative vein and a fondness for web design.
Currently, he uses tools like Adobe Creative Suite, Figma and TailwindCSS to design web pages.
\\[2pt]
Daniel loves to spend quality time with his wife and their five children.
When he turns the computer off, he rides one of his bikes or relaxes after going to the gym.
\end{adjustwidth}

\vspace{1em}
\section{\textsc{Professional Experience}}

\tlcventry{2014/3}{0}{Open Source Project Leader}{\href{https://github.com/vavr-io/vavr}{Vavr} (formerly Javaslang)}{}{}{
  \vspace{-5pt}
  \begin{itemize}\setlength\itemsep{4pt}
    \item Reified the vision of being \#1 project for functional programming in Java.
          Built a vibrant international community and a network of open source developers and sponsors.
    \item Designed an API for Haskell- and Scala-like algebraic data structures in Java (Functors, Applicatives, Monads).
          Implemented a complete set of persistent collections. Simplified concurrent programming.
    \item Wrote online documentation and components for code generation and automated testing
          on top cloud infrastructure.
  \end{itemize}
  \textbf{Keywords}: \textit{Vavr, Functional Programming, Java, Scala, GitHub, Open Source}
}

\tlcventry{2008/11}{2020/12}{Senior Software Engineer}{\href{https://www.hcob-bank.de/}{Hamburg Commercial Bank}}{Kiel/Hamburg}{}{
  \vspace{-5pt}
  (\textit{most relevant projects only})
  \begin{itemize}\setlength\itemsep{4pt}
    \item \textit{Microservice Integration Architecture for Capital Markets IT (2019-2020)} \\[2pt]
          \textbf{Role}: Lead Developer \\[2pt]
          Designed and implemented a Capital Markets data access layer based on Open API/spec first, incl. security.
          Implemented an OAuth2 server and adapted it to the in-house Corporate LDAP.
          Team collaboration creating services with Spring Boot including API gateway, service discovery and load balancing.
          Wrote deployment and release automation scripts for an in-house infrastructure.
          Wrote developer guidelines and shared the knowledge with my team.
          Adapted the services to existing systems in a hetorogenous landscape, like OneSumX and Cloudera. \\[2pt]
          \textbf{Keywords}: \textit{Oracle, SQL, Spring Boot, Java, Python, Shell, OneSumX, Cloudera}
    \item \textit{Digital transformation of the Know Your Customer onboarding process (2018-2019)} \\[2pt]
          \textbf{Role}: Lead Developer \\[2pt]
          Analyzed and designed the KYC onboarding process.
          Implemented an event-driven frontend in React and TypeScript that reflected updates of the onboarding process in real-time.
          Implemented a backend in Node.js with a CouchDB that concurrently collected private company data from 3rd party services like Bureau van Dijk/Orbis and the Germany Commercial Register.
          Trained junior developers to maintain and enhance the application. \\[2pt]
          \textbf{Keywords}: \textit{CouchDB, NoSQL, React, TypeScript, Node.js, JavaScript}
    \item \textit{Blockchain based trading platform for bonded loans (2016-2018)} \\[2pt]
          \textbf{Role}: Software Engineer \\[2pt]
          Worked with an interdisciplinary team of German banks on the architecture and implementation of a trading platform for bonded loans, called \href{https://www.der-bank-blog.de/schuldscheindarlehen-blockchain/technologie/37670219/}{finledger}.
          Solved backend data synchronization challenges in a distributed, eventually persistent architecture with private data portions per bank.
          Created a frontend with React, TypeScript and Ant Design.
          Implemented the backend with Spring Boot and the Ethereum blockchain. \\[2pt]
          \textbf{Keywords}: \textit{Ethereum, React, TypeScript, Node.js, Spring Boot}
    \item \textit{Migration of a pricing framework from C\# to Java (2015-2016)} \\[2pt]
          \textbf{Role}: Project Manager \\[2pt]
          Set up the project and designed a functional calculation core in plain Java.
          Trained the project team of financial engineers in functional programming in Java.
          Reimplemented with the team all pricers and adapted them to the Calypso trading system.
          Met the requirement of producing exactly same results while pricing big portions of the bank portfolio in a given time frame on a daily basis. \\[2pt]
          \textbf{Keywords}: \textit{C\#, .NET, Java, Calypso}
    \item \textit{Creation of a Calypso risk instance (2012-2015)} \\[2pt]
          \textbf{Role}: Software Engineer \\[2pt]
          Collaborated with a team of \char`~20 developers to created a risk instance of the Calypso trading system.
          My part was the design and implementation of an event bus for sharing end of day and neartime data with external systems. \\[2pt]
          \textbf{Keywords}: \textit{Calypso, Java, Enterprise Integration, FpML}
    \item \textit{Workflow automation of Credit Risk- \& Loan and Collateral Management (2011-2012)} \\[2pt]
          \textbf{Role}: Project Manager \\[2pt]
          Digitally transformed the workflow between two business units with IBM WebSphere Process Server on the backend
          and SAP WebDynpro for Java on the frontend. \\[2pt]
          \textbf{Keywords}: \textit{IBM WebSphere, SAP WebDynpro, Java}
  \end{itemize}
}

\tlcventry{2008/1}{2008/10}{Software Engineering Consultant}{\href{http://www.gentleware.com}{Gentleware AG}}{Hamburg}{}{
  \vspace{-5pt}
  \begin{itemize}\setlength\itemsep{4pt}
    \item HSH Nordbank AG: \textit{Middleware for booking final taxes (2008)} \\[2pt]
          \textbf{Role}: Software Engineer \\[2pt]
          Designed and implemented an enterprise middleware based on distributed transactional queues
          for passing payments and final taxes to multiple (external) systems. \\[2pt]
          \textbf{Keywords}: Weblogic Server, IBM MQ, Java EE
  \end{itemize}
}

\tlcventry{2002/8}{2007/12}{Software Developer}{\href{https://www.bminformatik.de/}{b+m Informatik AG}}{Kiel/Melsdorf}{}{
  \vspace{-5pt}
  \begin{itemize}\setlength\itemsep{4pt}
    \item HSH Nordbank AG: \textit{Web application for compliance (2007)} \\[2pt]
          \textbf{Role}: Software Engineer \\[2pt]
          Created a Java client/server application to support compliance. \\[2pt]
          \textbf{Keywords}: Weblogic Server, Java EE, JSF, Struts, Tiles
    \item UBS Germany: \textit{Reimplementation of the asset management system (2006)} \\[2pt]
          \textbf{Role}: Business Analyst \\[2pt]
          Consultancy, requirement analysis and reimplementation of the asset management system.
          Integration of business logic distributed over a Java application and Oracle PL/SQL. \\[2pt]
          \textbf{Keywords}: Business Process Optimization, Model Driven Software Development
    \item \textit{Creating a loan calculator (2006)} \\[2pt]
          \textbf{Role}: Software Developer \\[2pt]
          Created an amortization plan calculator in Java for different loan types.
          b+m is still selling it as a product. \\[2pt]
          \textbf{Keywords}: Loan Calculator, Java
    \item \textit{In-house development of financial products (2004-2006)} \\[2pt]
          \textbf{Role}: Software Developer \\[2pt]
          Developed products as part of a big team for the financial sector based on model driven software development and code generation.
          WebSphere and PostgreSQL on the backend, web browser on the frontend. \\[2pt]
          \textbf{Keywords}: MDSD, UML, WebSphere, Java, PostgreSQL
    \item \textit{Eclipse IDE plugin/language extension (2004)} \\[2pt]
          \textbf{Role}: Software Developer \\[2pt]
          Developed an Eclipse plugin for code generator languages that provided syntax highlighting, code formatting and code completion. \\[2pt]
          \textbf{Keywords}: Eclipse IDE, Plugin Development
  \end{itemize}
}

\tlcventry{2001/2}{2001/4}{Software Developer}{\href{https://www.mobilcom-debitel.de}{Mobilcom AG}}{Kiel}{}{
  \vspace{-5pt}
  \begin{itemize}\setlength\itemsep{4pt}
    \item Worked on the online portal, implemented data structures for hierarchically displaying information. \\[2pt]
          \textbf{Keywords}: JavaScript, Perl, LAMP stack (Linux, Apache, MySQL, PHP)
  \end{itemize}
}

\tlcventry{1998/5}{2002/9}{Student Assistant}{\href{https://www.uni-kiel.de/medinfo/institut/}{Institute of Medical Informatics and Statistics}}{Kiel}{}{
  \vspace{-5pt}
  \begin{itemize}\setlength\itemsep{4pt}
    \item Responsible for the computer lab.
          Gave introductory courses about biometrics for medical students.
    \item Used statistics software SPSS and R.
    \item Built a multi-user application based on MS Access.
  \end{itemize}
  \textbf{Keywords}: Statistics, SPSS, R, Access Database
}

\pagebreak

\section{\textsc{Technical Skills}}

\cvcomputer{Programming languages}{\begin{flushleft}TypeScript, JavaScript, CSS, HTML, Java, Scala, SQL, NoSQL, Bash, Python, PHP, Haskell\end{flushleft}}
           {Others}{\begin{flushleft}Git, Maven, Gradle, sbt, npm, yarn, Ethereum, GitHub Actions, Docker, Travis CI, Jenkins, Figma, Adobe Photoshop\end{flushleft}}
\cvcomputer{Development methodologies}{\begin{flushleft}Onion Architecture, Clean Code, GoF Design Patterns, SOLID, Agile Development, SCRUM, Domain-Driven Design (DDD), Test-Driven Development (TDD)\end{flushleft}}
           {Interest Areas}{\begin{flushleft}User Experience, Design, Functional Programming, Distributed Programming, Artificial Intelligence, Deep Learning, Blockchain, Kubernetes, DevOps\end{flushleft}}

% \cventry[spacing]{years}{degree/job title}{institution/employer}{localization}{optional: grade/...}{optional: comment/job description}

\section{\textsc{Articles and Books}}
\cventry{2018}{Article \href{https://jax.de/blog/core-java-jvm-languages/das-beste-aus-beiden-welten/}{"Das Beste aus beiden Welten - Objektfunktionale Programmierung mit Vavr"}}{\href{https://jaxenter.de/ausgaben/java-magazin-3-18}{Java magazin, 20th anniversary edition}}{}{Author of the month}{}
\cventry{2018}{Article \href{https://kiosk.entwickler.de/entwickler-magazin/entwickler-magazin-spezial-vol-18/web3j-blockchain-und-klimawandel/}{"Web3j, Blockchain und Klimawandel"}}{\href{https://kiosk.entwickler.de/entwickler-magazin/entwickler-magazin-spezial-vol-18/}{Entwickler Spezial Vol. 18}}{}{}{}
\cventry{2016}{Book reviewer \href{https://www.amazon.com/Reactive-Programming-Scala-Akka/dp/1783984341}{"Reactive Programming with Scala and Akka"}}{Packt Publishing}{}{}{}
\cventry{2013}{Book \href{https://www.amazon.com/Instant-Framework-Starter-Daniel-Dietrich/dp/1782162909}{"Instant Play Framework Starter"}}{Packt Publishing}{}{}{}

\section{\textsc{Courses and certifications}}
\cventry{2019}{\href{https://www.coursera.org/learn/neural-networks-deep-learning}{Neural Networks and Deep Learning}}{Coursera}{\href{https://coursera.org/share/f744684a1fa98c6377df85dd41e8ab6d}{Course certificate}}{}{}
\cventry{2015}{\href{https://www.edx.org/course/introduction-to-functional-programming}{Functional Programming in Haskell}}{Delft University}{\href{https://verify.edx.org/cert/c98eb8e2ed314fb98a3d9b1490560021}{Course certificate}}{}{}
\cventry{2014}{\href{https://www.mooc-list.com/course/principles-reactive-programming-coursera}{Principles of Reactive Programming}}{Coursera}{\href{https://www.coursera.org/api/legacyCertificates.v1/spark/statementOfAccomplishment/971465~707948/pdf}{Statement of Accomplishment}}{with distinction}{}
\cventry{2012}{\href{https://www.coursera.org/learn/progfun1}{Functional Programming Principles in Scala}}{Coursera}{\href{https://www.coursera.org/api/legacyCertificates.v1/spark/statementOfAccomplishment/308~707948/pdf}{Statement of Accomplishment}}{with distinction}{}
\cventry{2012}{Certified ScrumMaster}{Scrum Alliance, Inc.}{\href{http://bcert.me/sdvirlpn}{Course certificate}}{}{}

\section{\textsc{Education}}
\cventry{1997-2006}{Degree in Computer Science}{\href{https://www.uni-kiel.de/en/}{Christian-Albrecht University of Kiel}}{Kiel/Germany}{}{Final degree dissertation "Generator Debugging im modellgetriebenen Software-Entwicklungsprozess: Architektur und Implementierung"}

\section{\textsc{Languages}}
\cvlanguage{German}{Mother tongue}{}
\cvlanguage{English}{Fluent}{}

%\section{Driving license}
%\cvline{}{B license.}

%\section{Interests}
%\cvline{hobby 1}{\small Description}
%\cvline{hobby 2}{\small Description}
%\cvline{hobby 3}{\small Description}

\renewcommand{\listitemsymbol}{-} % change the symbol for lists

%\section{Extra 1}
%\cvlistitem{Item 1}
%\cvlistitem{Item 2}
%\cvlistitem[+]{Item 3}            % optional other symbol

%\section{Extra 2}
%\cvlistdoubleitem[\Neutral]{Item 1}{Item 4}
%\cvlistdoubleitem[\Neutral]{Item 2}{Item 5}
%\cvlistdoubleitem[\Neutral]{Item 3}{}

% Publications from a BibTeX file
%\nocite{*}
%\bibliographystyle{plain}
%\bibliography{publications}       % 'publications' is the name of a BibTeX file

\end{document}
